% ==========================================================
% Convex Shapes via Support Functions & Fourier — Research Book Template
% File: main.tex
% ==========================================================
\documentclass[11pt,oneside]{book}

% ---------- Packages ----------
\usepackage[margin=1in]{geometry}
\usepackage[T1]{fontenc}
\usepackage{lmodern}
\usepackage{microtype}
\usepackage{amsmath,amssymb,amsthm,mathtools}
\usepackage{enumitem}
\usepackage{graphicx}
\usepackage{xcolor}
\usepackage{hyperref}
\usepackage{cleveref}
\usepackage{booktabs}
\usepackage{float}
\usepackage{caption}
\usepackage{subcaption}
\usepackage{tikz}
\usepackage{listings}

\hypersetup{
  colorlinks=true,
  linkcolor=blue!60!black,
  citecolor=blue!60!black,
  urlcolor=blue!60!black
}

% ---------- Theorem-like environments ----------
\theoremstyle{plain}
\newtheorem{theorem}{Theorem}[chapter]
\newtheorem{lemma}[theorem]{Lemma}
\newtheorem{proposition}[theorem]{Proposition}
\newtheorem{corollary}[theorem]{Corollary}

\theoremstyle{definition}
\newtheorem{definition}{Definition}[chapter]
\newtheorem{example}{Example}[chapter]

\theoremstyle{remark}
\newtheorem*{remark}{Remark}

% ---------- Inquiry-based environments ----------
\newtheoremstyle{inquirystyle}
{8pt}{8pt}{}{}{\bfseries}{.}{.5em}{}
\theoremstyle{inquirystyle}
\newtheorem{inquiry}{Inquiry}[chapter]

\newtheoremstyle{taskstyle}
{6pt}{6pt}{}{}{\bfseries}{.}{.5em}{}
\theoremstyle{taskstyle}
\newtheorem{task}{Task}[chapter]

\newtheoremstyle{hintstyle}
{6pt}{6pt}{}{}{\itshape}{.}{.5em}{}
\theoremstyle{hintstyle}
\newtheorem{hint}{Hint}[chapter]

\newtheoremstyle{teststyle}
{6pt}{6pt}{}{}{\bfseries}{.}{.5em}{}
\theoremstyle{teststyle}
\newtheorem{test}{Test}[chapter]

% ---------- Short macros ----------
\newcommand{\R}{\mathbb{R}}
\newcommand{\Sone}{\mathbb{S}^1}
\newcommand{\ip}[2]{\langle #1,#2\rangle}
\newcommand{\norm}[1]{\lVert #1\rVert}
\newcommand{\dd}{\,\mathrm{d}}
\newcommand{\e}{\mathrm{e}}

% ---------- Code listing (Python) ----------
\lstdefinestyle{py}{
  language=Python,
  basicstyle=\ttfamily\small,
  keywordstyle=\color{blue!70!black},
  commentstyle=\color{green!40!black},
  stringstyle=\color{red!50!black},
  showstringspaces=false,
  frame=single,
  rulecolor=\color{black!15},
  breaklines=true,
  tabsize=4
}
\lstset{style=py}

% ---------- Document meta ----------
\title{Convex Shapes, Support Functions, and Fourier Representations\\
\large An Inquiry-Driven Research Book (Living Document)}
\author{Joel Amir Dario Maldonado Tänori}
\date{\today}

\begin{document}
\frontmatter
\maketitle
\tableofcontents

% ==========================================================
% Preface / How to use
% ==========================================================
\chapter*{How to Use This Book}
\addcontentsline{toc}{chapter}{How to Use This Book}

This is a living research book organized as a sequence of inquiries. Each chapter has:
(i) a small number of \textbf{Inquiries} (what we want to understand),
(ii) \textbf{Tasks} (proofs, computations, plots),
(iii) \textbf{Tests} (sanity checks / invariants),
(iv) a short \textbf{Synthesis} (what is now known, what remains uncertain).

\paragraph{Workflow.}
Write a short entry every time you (a) prove something, (b) run an experiment, (c) change a definition, or (d) update code.

\paragraph{Reproducibility.}
Every figure should have: data source, script path, parameters, and seed (if randomized).

% ==========================================================
% Notation
% ==========================================================
\chapter*{Notation and Conventions}
\addcontentsline{toc}{chapter}{Notation and Conventions}

\begin{itemize}[leftmargin=1.2em]
\item A convex body is $K\subset\R^2$ compact, convex, with nonempty interior (unless stated otherwise).
\item Unit direction $u(\theta) = (\cos\theta,\sin\theta)$, $\theta\in[0,2\pi)$.
\item Support function $h_K(\theta) = \sup_{x\in K} \ip{x}{u(\theta)}$.
\item Radial function (for star-shaped sets) $r(\theta)$ defined when appropriate.
\item Fourier series convention: for periodic $f(\theta)$,
\[
f(\theta)\approx \sum_{k=-K}^{K} c_k \e^{ik\theta},\quad
c_k = \frac{1}{2\pi}\int_0^{2\pi} f(\theta)\e^{-ik\theta}\dd\theta.
\]
\end{itemize}

\mainmatter

% ==========================================================
% Chapter 1
% ==========================================================
\chapter{Support Functions: The Dual Geometry}

\section{Core definitions}
\begin{definition}[Support function]
Let $K\subset\R^2$ be convex and compact. Its support function is
\[
h_K(\theta)=\max_{x\in K}\ip{x}{u(\theta)},\qquad u(\theta)=(\cos\theta,\sin\theta).
\]
\end{definition}

\begin{inquiry}[What does $h(\theta)$ measure?]
Explain geometrically what $h_K(\theta)$ represents (supporting line, directional extent), and why it is a natural object for packing/collision.
\end{inquiry}

\begin{task}[Translation identity]\label{task:translation-identity}
Prove that for any $v\in\R^2$,
\[
h_{K+v}(\theta) = h_K(\theta) + \ip{v}{u(\theta)}.
\]
\end{task}

\begin{hint}
Use the definition and substitute $y=x+v$.
\end{hint}

\begin{task}[Rotation identity]\label{task:rotation-identity}
Let $R_\phi$ be rotation by $\phi$. Prove:
\[
h_{R_\phi K}(\theta)=h_K(\theta-\phi).
\]
\end{task}

\begin{test}[Numerical sanity checks]\label{test:basic-invariances}
Implement a routine to sample $h_K(\theta)$ for a polygon and verify \Cref{task:translation-identity,task:rotation-identity} numerically to within tolerance.
\end{test}

\section{Synthesis}
\paragraph{What we now know.}
\begin{itemize}[leftmargin=1.2em]
\item Translation acts as adding a first-harmonic linear functional.
\item Rotation acts as phase shift on $\theta$.
\end{itemize}
\paragraph{What remains unclear.}
\begin{itemize}[leftmargin=1.2em]
\item Which geometric features are stable under Fourier truncation?
\item How convexity constraints appear as inequalities on $h$.
\end{itemize}

% ==========================================================
% Chapter 2
% ==========================================================
\chapter{Fourier Representation of Support Functions}

\section{Sampling and FFT pipeline}
\begin{inquiry}[What does smoothness mean spectrally?]
Relate regularity of $h(\theta)$ (smooth vs corners) to decay of Fourier coefficients.
\end{inquiry}

\begin{task}[FFT experiment template]\label{task:fft-template}
Pick a shape $K$ and produce:
(i) $K$ in physical space,
(ii) plot of $h(\theta)$,
(iii) log-magnitude spectrum $|c_k|$.
\end{task}

\begin{test}[Rotation as phase shift in Fourier coefficients]\label{test:phase-shift}
Rotate $K$ by $\phi$ and verify $c_k$ changes by factor $\e^{-ik\phi}$ (up to numerical error).
\end{test}

\section{Notes / results}
% Add short derivations, observed decay rates, and failure modes here.

% ==========================================================
% Chapter 3
% ==========================================================
\chapter{Teaching Set: Circle, Ellipse, Square, Regular Polygons}

\section{Circle}
\begin{inquiry}[Why is the circle spectrally sparse?]
Explain why only the $k=0$ mode remains for $h(\theta)\equiv R$.
\end{inquiry}

\section{Ellipse}
\begin{inquiry}[Anisotropy in Fourier space]
Explain the qualitative structure of Fourier coefficients for an ellipse and how it differs from a circle.
\end{inquiry}

\section{Square and corners}
\begin{inquiry}[Corners as high-frequency content]
Explain and illustrate Gibbs-like artifacts when truncating Fourier series of a square support function.
\end{inquiry}

% ==========================================================
% Chapter 4
% ==========================================================
\chapter{Constant Width, Radial Functions, and Convex Hull}

\section{Constant width}
\begin{inquiry}[Constant width constraint]
Show (or verify) the identity
\[
h(\theta)+h(\theta+\pi)=\text{constant}
\]
for constant-width shapes and interpret in Fourier terms.
\end{inquiry}

\section{Radial vs support}
\begin{inquiry}[What does the support function forget?]
Compare a nonconvex star-shaped example $r(\theta)=1+\varepsilon\cos(k\theta)$ with its convex hull and explain which features vanish in $h(\theta)$.
\end{inquiry}

% ==========================================================
% Chapter 5
% ==========================================================
\chapter{Toward Packing: Collision, Minkowski Sums, and Configuration Space}

\begin{inquiry}[Why support functions are good for packing]
Formulate collision/clearance in terms of support functions (e.g., via Minkowski difference / separating hyperplanes). Identify what becomes linear and what remains nonlinear.
\end{inquiry}

\begin{task}[Minkowski sum identity]
Prove $h_{A+B}(\theta)=h_A(\theta)+h_B(\theta)$ for convex compact sets $A,B$.
\end{task}

\begin{task}[Computational check]
Numerically verify Minkowski-sum additivity for polygons by sampling $h$.
\end{task}

% ==========================================================
% Chapter 6
% ==========================================================
\chapter{Application: Christmas Tree Shape}

\section{Tree geometry}
\begin{inquiry}[Tree in support-function space]
Compute $h(\theta)$ for the Christmas tree polygon and its convex hull. Compare spectra and discuss what geometric information is lost under convexification.
\end{inquiry}

\section{Band-limited approximations}
\begin{inquiry}[How many Fourier modes are enough?]
Define an accuracy metric (Hausdorff proxy, area error, max support error) and study how approximation quality scales with truncation level $K$.
\end{inquiry}

% ==========================================================
% Appendix: Reproducibility
% ==========================================================
\appendix
\chapter{Reproducibility Ledger}

\section{Figure checklist}
For each figure, record:
\begin{itemize}[leftmargin=1.2em]
\item Script path:
\item Data / parameters:
\item Random seed:
\item Output file path:
\end{itemize}

\section{Code snippets}
\begin{lstlisting}[caption={Support function sampling (placeholder)},label={lst:support-fn}]
# TODO: implement in your repo; include path and commit hash in ledger.
def support_function_polygon(poly, thetas):
    # return max_{v in vertices} <v, u(theta)> for each theta
    ...
\end{lstlisting}

\backmatter
\chapter*{Change Log}
\addcontentsline{toc}{chapter}{Change Log}
% Append brief dated notes about structural changes to the book.

\end{document}
